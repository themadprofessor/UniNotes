\documentclass[a4paper, 11pt]{article}
\usepackage[utf8]{inputenc}
\usepackage{mathtools}
\usepackage{gensymb}

\title{Exploring the Sky Notes}
\author{Stuart Reilly}
\date{\today}

\begin{document}
\maketitle

\section{Basic Observations}
What can we see in the sky?
\begin{enumerate}
	\item The Sun
		\begin{itemize}
			\item The big bright thing in the sky
			\item Don't look directly at the sun through a telescope, instead use white card placed below the eye piece.
			\item Provides heat and light for the Earth and all other bodies in the Solar System
			\item All most all energy on the Earth comes from the sun, the only exception being nuclear power
		\end{itemize}
	\item The Moon
		\begin{itemize}
			\item The big bright thing in the sky during the night
			\item Roughly a million times fainter than the Sun
			\item Its a natural satelite and our closet astronomical neighbour
			\item Only visible due the light from the Sun reflecting of its surface
			\item Can be seen during the day, but much fainter due the Sun overpowering the light from it
			\item Has phases due to the orbit of the Moon around the Earth
			\item Synchronous rotation (same face always faces the Earth) due to the tidal forces from the Earth
		\end{itemize}
	\item The Planets
		\begin{itemize}
			\item Fairly faint during the early morning and evening
			\item Only visible due to light from the Sun reflecting off the planet In or
			\item Some are easily spotted, and have been know since ancient times. In order of prominence:
				\begin{itemize}
					\item Venus
					\item Jupiter
					\item Mars
					\item Saturn
					\item Mercury
				\end{itemize}
			\item The rest were discovered using telescopes
				\begin{itemize}
					\item 1781 - Uranus recognised as a new planet
					\item 1846 - Neptune predicted and observed
					\item 1930 - Pluto found - now classed a dwarf planet
				\end{itemize}
			\item There are two ways to disinguish between a star and a planet:
				\begin{enumerate}
					\item Stars twinkle, planets don't
					\item Planets change position from night to night, hence planet in Greek means wanderer
				\end{enumerate}
			\item There are two types of planets:
				\begin{enumerate}
					\item Inferior - Mercury \& Venus
					\item Superior - Mars, Jupiter \& Saturn
				\end{enumerate}
			\item Mercury \& Venus are fairly close to the sun, making them only visible during the morning or evening
			\item Mercury is very difficult to see due to our latitude
		\end{itemize}
	\item The stars
		\begin{itemize}
			\item Looks like pin-point of light, even through large telescopes, due to the fecking enormous distance - the closest star is \(2.5\times2-^5\) further away than the Sun
			\item Twinkling caused by atmospheric irregularities in the star
			\item Some ''bright spark'' grouped them into constellations, which appear to have been unchanged over thousands of years
				\begin{itemize}
					\item Most come from Greek mythology - i.e. Orion = the mighty hunter - yeah, I don't see it either
					\item There are 88 constellations - yay
				\end{itemize}
			\item Not all stars have the smae brightness or colour, this is due to the size, temperature and distance of the star
		\end{itemize}
	\item The Milky Way
		\begin{itemize}
			\item Hard to see in populated areas due to light polution
			\item Looks like a diffuse band of light across the sky
			\item Telescopes can be used to pick out the stars
			\item Roughly \(2\times10^{11}\) stars
			\item Lens shaped due to buldge at the centre of the galaxy
		\end{itemize}
	\item Nebulae
		\begin{itemize}
			\item Latin for cloud
			\item Covers everyting outisde of the solar system which is not start like
				\begin{itemize}
					\item Stellar remnant - Crab nebula
					\item Hot gas clouds - Orion nebula
					\item External galaxies - Andromeda nebula
				\end{itemize}
		\end{itemize}
	\item Asteroids
		\begin{itemize}
			\item First one discovered in 1801, called Ceres - was thought to be a missing planet between Mars and Jupiter
			\item Many more discovered, called minor planets/asteroids
			\item Can only be seen through a telescope - telescopic object
			\item Pretty much only found in the asteroid belt
			\item There is a metric fuckton of small ones in there - irregular in shape due to limited gravity to create a rounded shape
			\item There is another asteroid belt past Neptune (Kuiper Belt), which is one reason why Pluto was demoted to dwarf plane
		\end{itemize}
	\item Artificial satellites
		\begin{itemize}
			\item Some "stars" appear to drift
			\item They are fairly high artifical satellites which reflect sunlight, with varing brightness as they rotate
			\item They suddenly disappear if they move into the Earth's shadow
		\end{itemize}
	\item Comets
		\begin{itemize}
			\item Bright object which travels from the outer solar system to the Sun
			\item They are only seen once, except Halley's comet which has a period of 76 years
			\item As they approach the Sun, the become brighter and develop a tail as they heat up
			\item They appear stationary in the sky and are visible for many nights
		\end{itemize}
	\item Meteors
		\begin{itemize}
			\item Objects which fall into the Earth's atmosphere and burn up, causing them to be quite bright and have a tail
			\item Can break up during their decent, causing a meteor shower
		\end{itemize}
	\item Aurorae
		\begin{itemize}
			\item Glow from particals in the high atmosphere, which have been stimulated by collsions from charged particals from space
			\item Best seen at high latitudes
			\item Best known is the Aurora Borealis - Northen Lights
			\item Another is the Aurora Australis
		\end{itemize}
	\item The Zodiacal Light
		\begin{itemize}
			\item Faint glow along the zodiac near the Sun
			\item Best seen along the equator, as twilight is the shortest there
		\end{itemize}
\end{enumerate}

From the list above, the Sun, Moon, Planets, Stars, Milky Way, Nebulae and Asteroids are regulary there to be seen whereas the rest are more unpredictable.

All astronomical bodies undergo diurnal motion. As is fairly obvious, the Sun moves across the sun from east to west, this is diurnal motion. This is caused by the rotation of the Earth, hense all astronomical bodies undergo this motion.

The Moon has a series of phases, based on its position within its orbit around the Earth, corrisponding to the amount of the moon which is illiminated by the sun and is facing the Earth. The cycle begins with the new moon, where the Earth-facing hemisphere is facing away from the Sun, and therefore is black. As the Moon rotates around its orbit, more of the Earth-facing hemisphere becomes illuminated, until the full moon, where the entire hemisphere is illuminated. Then, the proportion of the hemisphere which is illuminated decreases, until it reachs the new moon once more. When the Moon is a thin crescent, the Earth will appear fully illuminated from the Moon, earthshine.

\section{The Celestial Sphere}

\begin{itemize}
\item Superficial impression that all objects in the sky are stuck to the inside of a sphere
\item Lead the the geocentric model of the Universe
\item Only half of the sphere is visible at any time
\item No visible parallax on ''sphere of fixed stars''
	\begin{itemize}
	\item Sphere must be feckin huge
	\item Bolstered geocentric model
	\end{itemize}
\item The stars do show a tiny (fraction of degree) of parallax shift, so not seen till mid 19th century
\end{itemize}

\section{Dirunal Motion}
\begin{itemize}
\item Obviously, the Sun moved east to west during the course of a day
\item The stars, moon and planets also do the same (only visible during the night)
\item If one observes the sky at night, one can see three types of paths:
	\begin{itemize}
	\item Most rise in east, reach max height above horizon along north-south line (observer's maridian) and set symmetrically in the west
	\item Further east of south star rises, the higher its max hight above southern horizon
	\item Some stars don't set (circumpolar stars), also some stars never rise
	\end{itemize}
\item This apparent motion = diurnal motion = daily motion
\item Caused by rotation of the Earth
\item Causes celestial sphere to appear to rotate
\end{itemize}

\section{Finding Pole Star}
\begin{itemize}
\item Find The Plough, which is part of Ursa Major (the Great Bear)
\item These stars appear circumpolar in Glasgow, so are always visible
\item The ''cutting edge'' of The Plough points to the Pole Star, which is close to Ursa Minor (the Little Bear)
\end{itemize}

\section{The Measurement Of Angles}
\begin{itemize}
\item Hard to know how far shit is buy just staring at it, isn't it. Go on try. Tell me how far that one is.
\item Key angle facts:
	\begin{itemize}
	\item Full circle = \(360\degree\)
	\item Right angle = \(90\degree\)
	\item One degree = 60 arc minutes = 60'
	\item One arch minute = 60 arc seconds = 60''
	\end{itemize}
\item Note: arc seconds are fecking tiny, and there are 3600 in 1 degree like there are 3600 seconds in a minute
\item There are some helpful rules of thumb for estimating angles:
	\begin{itemize}
	\item Thickness of thumb/index finger at arms length \(\approx 1\degree\)
	\item Thickness of fist at arms length \(\approx 10\degree\)
	\item Thickness of outstretched hand at arms length \(\approx 1\degree\)
	\end{itemize}
\item A full moon is roughly half a degree, which means the sun is also roughly half a degree
\item The Polar Star is roughly \(56\degree\) above the north point
\end{itemize}

\section{Latitude and Longitude}
\begin{itemize}
\item Earth nearly perfect sphere
\item Shortest distance between two points on the Earth is a great circle
\item Great circle - any plane section through the centre of the Earth will intersect the surface in a circle, whose radius is equal to the radius of the Earth
\item Small circle - any plane section not through the centre of the Earth intersects the surface in a circle, whose radius is less than the radius of the Earth
\item Positions on the Earth are given by their longitude and latitude
\item North pole has latitude \(+90\degree\)
\item South pole has latitude \(-90\degree\)
\item Parallels of latitude = small circles of constat latitude parallel to the equator
\item Meridians of longitude = semi great circles at right angles to parallels of latitude and terminiating at the poles
\item \(0\degree\) latitude = equator, obviously
\item \(0\degree\) longitude = arbirary, but we chose Greenwich meridian
\end{itemize}

\section{Positional Effect on Celestial Sphere}
	\subsection{Latitude}
	\begin{itemize}
	\item Altitude of north celestial pole = latitude of the observer
	\item Glasgow's latitude \(\approx56\degree\), hence the Polar Star is \(\approx\degree\) above the north horizon as stated above
	\item At the north pole (latitude = \(90\degree\)), the Polar Star is directly above or \(90\degree\) above the horizon
	\item Stars wouldn't set, but travel in a circle
	\end{itemize}

	\subsection{Longitude}
		\begin{itemize}
			\item Changes when you see the sky, not which sky
			\item Greenwich and Iowa City have the same latitude, therefore will see the same section of the celstrial sphere
			\item But, they have differnet longitude and therefore will see the section at different times
			\item Since, Iowa City's longitude is \(91.5\degree\), the Earth needs to rotate \(91.5\degree\) for them to sse the same section
			\item Time for some conversions :)
				\begin{itemize}
				\item Since the Earth takes 24 hours to rotate \(360\degree\)
				\item \(360\degree\) = 24\textsuperscript{h}
				\item \(15\degree\) = 1\textsuperscript{h}
				\item 15' = 1\textsuperscript{m}
				\item 15'' = 1\textsuperscript{s}
				\end{itemize}
			\item Remember the \textit{m} and \textit{s} are only for time measures and the dashes are for angular measures
		\end{itemize}

\section{The Month}
	\begin{itemize}
		\item Number of months appear arbitrary
		\item Multiple astronomical definition for month:
			\begin{itemize}
			\item Sidereal Month - One full orbit of the moon - 27.32 days
			\item Synodic Month - Interval between each new moon - 29.53 days
			\item Difference caused by Earth's orbit - Moon travels bit further to reach original position
			\end{itemize}
		\item The synodic month used to base calender as easier to accurately measure from Earth
		\item Islamic calender has 12 months of 29/30 days - close to moon phases but year off my 11 days - no fixed seasons
		\item Jewish calender linked to moon phases - adds extra month periodically to align seasons
		\item Current calender - no links to moon - length is arbitrary - Feb has leap day - based of sun's appearance so solar calender
	\end{itemize}

\section{The Julian \& Gregorian Calenders}
	\begin{itemize}
		\item First solar calender - ancient Egypt - 12 months of 30 days + 5 festival days
		\item Year is 365.25 days - error creep in
		\item So add 6th festival day every 4 years
		\item Roman calender allowed for months to be added as needed
		\item Abused political - fell out of sync with seasons even though was solar calender
		\item Julian calender created
		\begin{itemize}
			\item 365 days in a year
			\item 366 days every 4 years
		\end{itemize}
		\item All good until 1582 - error became 20 days
		\item Found Earth's orbital period is 365.2564 days - sidereal year
		\item Equinox should be around March 21st - 365.2422 days between equinox's - tropical year
		\item Gregorian calender created in 1582 by Pope Gregory XIII
		\begin{itemize}
			\item 365 days per year
			\item 366 days every 4 years - leap year
			\item Not leap year if year is divisible by 4
		\end{itemize}
	\end{itemize}

\section{Distance to the Moon}
	\begin{itemize}
		\item 384,400 km
		\item Close enough to measure its geocentric parallax 
		\item If moon was observed from either side of the Earth simultaneously, it will appear to move \(2\degree\) relative to stellar background
		\item Stars cannot produce similar measurements with this technique due to their distance
		\item Even Venus would move less than 1 arc minute
		\item Now we use the laser reflectors left on the Moon by the Apollo missions to measure the distance, using \(speed = \frac{distance}{time}\)
	\end{itemize}

\section{The Tides}
	\begin{itemize}
		\item \(gravitation attriaction \propto (distance between the objects)^{-2}\)
		\item Due to this, and the mass of the Sun, the Sun's gravity dominates the Earth, but the Moon causes the Earth's orbit to have a wobble
		\item Tidal forces
		\begin{itemize}
			\item \(gravitation attriaction \propto (distance between the objects)^{-3}\)
			\item Hence, the Moon's tidal force is double the Sun's on the Earth
			\item Differential force
			\item Causes Earth to bulge
			\item Side nearer to Moon feels greater force
			\item Tides rise and fall due to Earth's rotation and Moon's orbit - interval of \(12^h 25^m\)
			\item Sun's tidal force is not negligable
			\item Spring tides (largest), happens at full and new moons as the Sun's tidal forces complement the Moon's
			\item Neap tides, happens at the half moons, as the Sun's tidal forces act against the Moon's
			\item The Earth exerts a tidal force on the Moon, causing it to be tidally locked - same face always facing the Earth
		\end{itemize}
	\end{itemize}

\section{The Orbit of the Moon}
	\begin{itemize}
		\item Anticlockwise, when seen from north
		\item One circuit takes a sidereal month - 27.23 days
		\item Each night, moves eastward \(13\degree\)
		\item \(5\degree\) off the ecliptic
		\item Intersects ecliptic at two diametrically opposite nodes
		\item Earth's orbit around Sun defines ecliptic
		\item Nodes regress around ecliptic once every 18.6 years
	\end{itemize}

\section{Eclipses of the Sun}
	\begin{itemize}
		\item Moon moves \(13\degree\) each day
		\item Sun moves \(1\degree\) each day
		\item New moon happens when Moon overtakes the Sun
		\item Would be solar eclipse if Moon orbit not \(5\degree\) off ecliptic
		\item Only happens if new moon near one of the nodes defined above
		\item Since nodes moving, Sun successfully pass particular node every 346 days - eclipse year
		\item Solar eclipse happen every 173 days - only visible in specific parts of Earth - normally partial eclipse
		\item Types of eclipse:
		\begin{itemize}
			\item Total - Earth must be within Moon's central shadow (Umbra)
			\item Partial - Earth lies in Moon's outer shadow (Penumbra)
			\item Annular - Earth directly behind Umbra - ring of Sun visible around Moon
		\end{itemize}
		\item Zone of totality of total eclipse few hundred km wide
		\item Caused by the Sun and Moon having same angular size
		\item Due to differences in orbital eccentricities of the Earth and Moon, angular sizes of Moon and Sun change
		\item If Sun's angular size larger than Moon's, total eclipse cannot happen but allows annular
	\end{itemize}

\end{document}
