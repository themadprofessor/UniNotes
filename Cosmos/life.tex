\documentclass[12pt]{article}
\author{Stuart Reilly}
\date{\today}
\title{Life in the Cosmos Notes}

\begin{document}
\maketitle

\section{Origin of elements, the solar system, the Earth's atmosphere and life's building blocks}
	\subsection{Origin of elements}
	The Earth is mostly made of:
	\begin{enumerate}
		\item Iron
		\item Silicon
		\item Nickel
		\item Magnesium
		\item Oxygen
	\end{enumerate}
	The Earth's atmosphere is mostly made of:
	\begin{enumerate}
		\item Nitrogen
		\item Oxygen
		\item Carbon Dioxide
	\end{enumerate}
	Life of Earth is mostly made of:
	\begin{enumerate}
		\item Carbon
		\item Oxygen
		\item Hydrogen
		\item Trace amounts of iron and phosphorus
	\end{enumerate}
	Only hydrogen is  a primordial element (originating before star formation begun).
	
	\subection{Nucleosynthesis}
	Nucleosynthesis is the building of nuclei from lighter nuclei.
	This only occurs within nuclear fusion reactors and the cores of stars.
	Stars use nucleosynthesis to fuel themselves, as it is an exothermic reaction.
	It requires extreme temperatures and pressures, which only exist for microseconds within a reactor or billions of year in the core of a star.
	The elements and the amount of them is dependent on the star's mass.
	A star's mass is proportional to the rate it uses it fuel and its core's temperature.
	Hence, not all stars will produce every element.
	The star produces layers of the elements it has created from nucleosynthesis.
	Once it begins to react using iron, its energy produces stops, as nucleosynthesis becomes endothermic.
	The radiation pressure from the core begins to weaken, causing the star to collapse under its own weight.
	Once the outer layers hit the core, they have enough force to rebound, creating a supernova explosion.
	The supernova explosion disperses the elements produced by the star into interstellar space.
	Due to the extreme forces within a supernova explosion, the remaining elements are produced.
	All of the materials found on the earth have been created within a star's life or its death.
	
	\subsection{Formation of the Solar System}
	Stars and planets from within nebula, which are areas of space with a high density of methane, water, methanol, ammonia, dust and gas.
	The dust is critical to the formation of the planets, where as the gas is critical to the formation of the Sun and the Jovian planets, the complex molecules are intrinsic to the formation of life.
	Complex molecules can be identified from their emission spectra, which a subset of the full electromagnetic spectrum, emitted by a molecule, which shows the elements its comprised of.
	Proto-stars are stars which have not begun nuclear burning.
	They and proto-planetary disks can be observed by their microwave and infra-red emission spectra.
	Being cool objects, their emission spectra is a longer wavelengths.
	Stars condense as gas and dust is pulled together due to their own gravities, creating a gradually denser sphere, which begins to rotate as a higher rate.
	As the star rotates faster, the particles outside of the disk begin to orbit rather than fall towards the star, forming another disk which lasts around 30 million years.

	\subsection{Stages of planetary formation}
	\paragraph{Condensation}
		Larger dust molecules act like condensation centres for smaller dust particles to stick to.
		This is very similar to how raindrops are formed.
	\paragraph{Accretion}
		The condensed particles collide together, forming planetesimals.
		They range from a few centimetres to kilometres across, but we are unsure how they stick together due their speed.
	\paragraph{Coalescence}
		The newly formed planetesimals collide together to form protoplanets.
		As these protoplanets grow in mass, they begin to develop a gravitational field strong enough to pull in more material, increasing their rate of growth.
	\\
	\\
	As the planets started to accrete, the Sun began to produce strong solar winds.
	This removed the gas from the inner disk, meaning early atmospheres where formed by chemical breakdown of planetesimals or outgassing from the planets.

	\subsection{The Hadean Era}
	
